% vim: nu expandtab shiftwidth=2 softtabstop=2 autoindent foldmethod=marker
%%%%%%%%%%%%%%%%%%%%%%%%%%%%%%%%%%%%%%%%%%%%%%%%%%%%%%%%%%%%%%%%%%%%%%%%%%%%%
% Consecutive Integers Explanation
%%%%%%%%%%%%%%%%%%%%%%%%%%%%%%%%%%%%%%%%%%%%%%%%%%%%%%%%%%%%%%%%%%%%%%%%%%%%%
\section{Consecutive Integers - Explanation}
Consecutive integers are positive or negative whole numbers that lie next to each other on a number line. 
The easiest example of consecutive integers are the numbers 1, 2, \& 3.

\begin{enumerate}
%%%% 1  %%%%
\item There are three consecutive integers whose sum is 90. What are the three integers?
  \begin{itemize}
  \item The first step is to set up an equation in terms of x. Lets say that the first term is x. Well, since the first term is x, we need to modify it to get to the second term. Since the second term is the next consecutive integer, it is one away on the number line, and so I will add one to x. So the second term is $x+1$. The third term is one more than the second, and so it is $x+1+1$ or $x+2$.
  \item [] $x+(x+1)+(x+2)=90$
  \item Now we simplify the equation and solve for x.
  \item [] $3x+3=90$
  \item [] $3x=87$
  \item [] $x=29$
  \item So the first integer is 29. Now we need to find the next two integers. The second integer we said is one more than the first, so it is $29+1$, or 30. The third integer we said was the first plus two, so it is $29+2$, or 31
  \item So the three consecutive integers that add up to 90 are 29, 30, and 31.
  \end{itemize}
%%%% 2  %%%%
\item There are four consecutive odd integers whose sum is 96. What are the four integers?
  \begin{itemize}
  \item The first step is to set up an equation in terms of x. Lets say that the first term is x. Well, since the first term is x, we need to modify it to get to the second term. Since the second term is the next consecutive odd integer, it is two away on the number line, and so I will add two to x. So the second term is $x+2$. The third term is two more than the second, and so it is $x+2+2$ or $x+4$. The fourth term is 2 more than the third term, and so it is $x+4+2$ or $x+6$
  \item [] $x+(x+2)+(x+4)+(x+6)=96$
  \item Now we simplify the equation and solve for x.
  \item [] $4x+12=96$
  \item [] $4x=84$
  \item [] $x=21$
  \item So the first integer is 21. Now we need to find the next three integers. The second integer we said is two more than the first, so it is $21+2$, or 23. The third integer we said was the first plus four, so it is $21+4$, or 25. The fourth integer we said was the first plus six, so it is $21+6$ or 27.
  \item So the four consecutive odd integers that add up to 96 are 21, 23, 25, and 27.
  \end{itemize}
%%%% 3  %%%%
\item There are two consecutive even integers whose sum is 10. What are the two integers?
  \begin{itemize}
  \item The first step is to set up an equation in terms of x. Lets say that the first term is x. Well, since the first term is x, we need to modify it to get to the second term. Since the second term is the next consecutive even integer, it is two away on the number line, and so I will add two to x. So the second term is $x+2$.
  \item [] $x+(x+2)=10$
  \item Now we simplify the equation and solve for x.
  \item [] $2x+2=10$
  \item [] $2x=8$
  \item [] $x=4$
  \item So the first integer is 4. Now we need to find the next integer. The second integer we said is two more than the first, so it is $4+2$, or 6.
  \item So the two consecutive even integers that add up to 10 are 4 and 6.
  \end{itemize}
\end{enumerate}
